%# -*- coding: utf-8-unix -*-
%%==================================================
%% thesis.tex
%%==================================================

% 双面打印
\documentclass[bachelor, fontset=adobe, openany, oneside, zihao=5, submit]{sjtuthesis}
% \documentclass[master, adobefonts, review]{sjtuthesis} 
% \documentclass[%
%   bachelor|master|doctor,	% 必选项
%   fontset=adobe|windows,  	% 只测试了adobe
%   oneside|twoside,		% 单面打印,双面打印(奇偶页交换页边距,默认)
%   openany|openright, 		% 可以在奇数或者偶数页开新章|只在奇数页开新章(默认)
%   zihao=-4|5,, 		% 正文字号:小四、五号(默认)
%   review,	 		% 盲审论文,隐去作者姓名、学号、导师姓名、致谢、发表论文和参与的项目
%   submit			% 定稿提交的论文,插入签名扫描版的原创性声明、授权声明 
% ]

\makeatletter
\patchcmd{\chapter}{\if@openright\cleardoublepage\else\clearpage\fi}{}{}{}
\makeatother


% 逐个导入参考文献数据库
\addbibresource{bib/thesis.bib}
% \addbibresource{bib/chap2.bib}

\begin{document}

%% 封面
%# -*- coding: utf-8-unix -*-
\projectinfo(十,IPP)
\projectno{8888}
\projecthead{王三}
\projectheadmajor{奇奇系}
\title{奇奇怪怪的东西的研究}

\advisor{某某教授}
\projectadvisormajor{怪怪系}

\projectstudents{王三,李四,张五}
\projecttime(2014,10,2015,9)
\defenddate{2014年12月17日}
\school{上海交通大学}
\institute{某某系}
\studentnumber{0010900990}
% --- below is abandoned
\author{某\quad{}某}
\major{某某专业}

\englishtitle{A Sample Document for \LaTeX-basedd SJTU Thesis Template}
\englishauthor{\textsc{Mo Mo}}
\englishadvisor{Prof. \textsc{Mou Mou}}
% \englishcoadvisor{Prof. \textsc{Uom Uom}}
\englishschool{Shanghai Jiao Tong University}
\englishinstitute{\textsc{Depart of XXX, School of XXX} \\
  \textsc{Shanghai Jiao Tong University} \\
  \textsc{Shanghai, P.R.China}}
\englishmajor{A Very Important Major}
\englishdate{Dec. 17th, 2014}


\makecover


\frontmatter 	% 使用罗马数字对前言编号

%% 摘要
\pagestyle{main}
\input{tex/abstract}

\mainmatter	% 使用阿拉伯数字对正文编号

%% 正文内容

%论文的全部标题层次应有条不紊,整齐清晰。正文内容编号方法应采用分级阿拉伯数字编号方法,
%第一级为“1”、“2”、“3”、等, chapter
%第二级为“2.1”、“2.2”、“2.3”等, section
%第三级为“2.2.1”、“2.2.2”、“2.2.3”等,subsection
%两级之间用下角圆点隔开,除第一级外,其余各级的末尾不加标点。
%各级标题均单独占行空两格书写序数,之后空一格书写标题。
%第一级标题小四号黑体,段前0.5行,段后0.5行,
%其余各级标题均为五号字体。
\pagestyle{main}
%正文字数8000-10000字左右,主要包括绪论、研究内容及方法、研究结果及讨论、结论等部分。
\input{tex/chapter01}
\input{tex/chapter02}
%# -*- coding: utf-8-unix -*-
\chapter{研究结果}

\section{\TeX}
\TeX (希腊语:/tɛx/[1],音译“泰赫”,文本模式下写作TeX),是一个由美国计算机教授高德纳(Donald Ervin Knuth)编写的功能强大的排版软件。它在学术界十分流行,特别是数学、物理学和计算机科学界。\TeX 被普遍认为是一个优秀的排版工具,特别是在处理复杂的数学公式时。利用诸如是\LaTeX 等终端软件,\TeX 就能够排版出精美的文本。

A:这份模板以Apache License 2.0开源许可证发布,请遵循许可证规范。

\section{\LaTeX}
\LaTeX(英语发音:/ˈleɪtɛk/ LAY-tek或英语发音:/ˈlɑːtɛk/ LAH-tek,音译“拉泰赫”),文字形式写作\LaTeX,是一种基于\TeX 的排版系统,由美国电脑学家莱斯利·兰伯特在20世纪80年代初期开发,利用这种格式,即使使用者没有排版和程序设计的知识也可以充分发挥由\TeX 所提供的强大功能,能在几天,甚至几小时内生成很多具有书籍品質的印刷品。对于生成复杂表格和数学公式,这一点表现得尤为突出。因此它非常适用于生成高印刷质量的科技和数学类文档。这个系统同样适用于生成从简单的信件到完整书籍的所有其他种类的文档。

\LaTeX 使用\TeX 作为它的格式化引擎,当前的版本是LaTeX2ε。


%# -*- coding: utf-8-unix -*-

\chapter{讨论}

\TeX 是非常稳定的程序,高德纳悬赏奖励任何能够在\TeX 中发现程序漏洞(bug)的人。每一个漏洞的奖励金额从1美分开始,并每年翻倍,直到目前的327.68美元封顶。然而高德纳从未因此而损失大笔金钱,因为\TeX 中的漏洞极少,而真正发现漏洞的人在获得支票后往往不愿将其兑现。

到目前为止,关于\TeX 的最后一个bug是被Oleg Bulatov发现的。
\input{tex/summary}

%\appendix	% 使用英文字母对附录编号,重新定义附录中的公式、图图表编号样式
%\renewcommand\theequation{\Alph{chapter}--\arabic{equation}}	
%\renewcommand\thefigure{\Alph{chapter}--\arabic{figure}}
%\renewcommand\thetable{\Alph{chapter}--\arabic{table}}
%\renewcommand\thealgorithm{\Alph{chapter}--\arabic{algorithm}}

%% 附录内容,本科学位论文可以用翻译的文献替代。
%\include{tex/app_setup}
%\include{tex/app_eq}
%\include{tex/app_cjk}
%\include{tex/app_log}

%% 参考资料
\printbibliography[heading=bibintoc]

%% 致谢、发表论文、申请专利、参与项目、简历
%% 用于盲审的论文需隐去致谢、发表论文、申请专利、参与的项目
\makeatletter
\ifsjtu@review\relax\else
  \input{tex/ack} 	  %% 致谢
%  \include{tex/pub}	  %% 发表论文
%  \include{tex/patents}	  %% 申请专利
%  \include{tex/projects}  %% 参与的项目
% \include{tex/resume}	  %% 各人简历
\fi
\makeatother

\end{document}
